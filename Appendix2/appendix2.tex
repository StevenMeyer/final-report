\chapter{Code samples}

\section{CoffeeScript Class --- Table}

This class is an example of using CoffeeScript for OOP. This example contains the standard pseudo-classical pattern for public properties and the all-in-one constructor pattern for private and privileged properties.

This example has been subjected to line wrapping. Whitespace in Coffee\-Script is significant and, unfortunately, the wrapping here would not constitute well-formed Coffee\-Script code.

\begin{lstlisting}
class Table
  @IdMethod =
    NATIVE:  "native"
    NONE:    "none"
    toArray: () ->
      this[method] for method of this when method isnt "toArray"
      
  @INHERIT = "__inherit__"
      
  @PhpNamingMethod =
    CLEAN:      "clean"
    NOCHANGE:   "nochange"
    PHPNAME:    "phpname"
    UNDERSCORE: "underscore"
    toArray:    () ->
      this[method] for method of this when method isnt "toArray"
      
  @TreeMode =
    MATERIALIZED_PATH: "materializedPath"
    NESTED_SET:        "nestedSet"
    toArray:           () ->
      this[mode] for mode of this when mode isnt "toArray"
      
  constructor: (name) ->
    _attributes =
      abstract: false
      allowPkInsert: false
      baseClass: Table.INHERIT
      basePeer: Table.INHERIT
      description: null
      heavyIndexing: Table.INHERIT
      idMethod: Table.INHERIT
      isCrossRef: false
      name: null
      namespace: Table.INHERIT
      package: Table.INHERIT
      phpName: null
      phpNamingMethod: Table.INHERIT
      readOnly: false
      reloadOnInsert: false
      reloadOnUpdate: false
      schema: Table.INHERIT
      skipSql: false
      treeMode: null
    _columns = new LinkedList()
    _foreignKeys = new LinkedList()
    _indices = new LinkedList()
    _uniqueIndices = new LinkedList()
    
    if not String::trim
      String::trim = () ->
        this.replace(/^\s+|\s+$/g, '')
    
    @addColumn = (column) ->
      if column instanceof Column
        _columns.addItem column
      else
        false
        
    @addColumnBefore = (column, before) ->
      if column instanceof Column
        _columns.addItem column, before
      else
        false
        
    @addForeignKey = (foreignKey) ->
      if foreignKey instanceof ForeignKey
        _foreignKeys.addItem(foreignKey)
      else
        false
        
    @addIndex = (index) ->
      if index instanceof Index
        _indices.addItem(index)
      else
        false
        
    @addUniqueIndex = (uniqueIndex) ->
      if uniqueIndex instanceof UniqueIndex
        _uniqueIndices.addItem(uniqueIndex)
      else
        false
    
    @allowPkInsert = () -> _attributes.allowPkInsert
    
    @getBaseClass = () -> _attributes.baseClass
    
    @getBasePeer = () -> _attributes.basePeer
    
    @getDescription = () -> _attributes.description
    
    @getIdMethod = () -> _attributes.idMethod
    
    @getColumns = () -> _columns.getItems()
    
    @getForeignKeys = () -> _foreignKeys.getItems()
    
    @getIndices = () -> _indices.getItems()
    
    @getUniqueIndices = () -> _uniqueIndices.getItems()
    
    @getName = () -> _attributes.name
    
    @getNamespace = () -> _attributes.namespace
    
    @getPackage = () -> _attributes.package
    
    @getPhpName = () -> _attributes.phpName
    
    @getPhpNamingMethod = () -> _attributes.phpNamingMethod
    
    @getSchema = () -> _attributes.schema
    
    @getTreeMode = () -> _attributes.treeMode
    
    @isAbstract = () -> _attributes.abstract
    
    @isCrossRef = () -> _attributes.isCrossRef
    
    @isHeavyIndexing = () -> _attributes.heavyIndexing
    
    @isReadOnly = () -> _attributes.readOnly
    
    @isSkipSql = () -> _attributes.skipSql
    
    @reloadOnInsert = () -> _attributes.reloadOnInsert
    
    @reloadOnUpdate = () -> _attributes.reloadOnUpdate
    
    @removeColumn = (column) ->
      if column instanceof Column
        _columns.removeItem(column)
      else false
      
    @removeForeignKey = (foreignKey) ->
      if foreignKey instanceof ForeignKey
        _foreignKeys.removeItem(foreignKey)
      else
        false
        
    @removeIndex = (index) ->
      if index instanceof Index
        _indices.removeItem(index)
      else
        false
        
    @removeUniqueIndex = (uniqueIndex) ->
      if uniqueIndex instanceof UniqueIndex
        _uniqueIndices.removeItem(uniqueIndex)
      else
        false
      
    @setAbstract = (bool = true) ->if bool
      _attributes.abstract = if bool then true else false
      
    @setAllowPkInsert = (bool = true) ->
      _attributes.allowPkInsert = if bool then true else false
      
    @setBaseClass = (baseClass) ->
      baseClass = baseClass.trim() if typeof baseClass is "string"
      _attributes.baseClass = if baseClass then baseClass else Table.INHERIT
      
    @setBaseClass = (basePeer) ->
      basePeer = basePeer.trim() if typeof basePeer is "string"
      _attributes.basePeer = if basePeer then basePeer else Table.INHERIT
      
    @setDescription = (description) ->
      _attributes.description = if description then description else null
      
    @setHeavyIndexing = (bool = true) ->
      if bool is Table.INHERIT then _attributes.heavyIndexing = Table.INHERIT
      _attributes.heavyIndexing = if bool then true else false
      
    @setIdMethod = (method = Table.INHERIT) ->
      if method in Table.IdMethod.toArray() then _attributes.method =  method else Table.INHERIT
      
    @setIsCrossRef = (bool = true) ->
      _attributes.isCrossRef = if bool then true else false
    
    @setName = (name) ->
      name = name.trim() if typeof name is "string"
      _attributes.name = if name then name else throw "Table must have a name"
      
    @setNamespace = (namespace) ->
      namespace = namespace.trim if typeof namespace is "string"
      _attributes.namespace = if namespace then namespace else Table.INHERIT
      
    @setPackage = (thepackage) ->
      thepackage = thepackage.trim if typeof thepackage is "string"
      _attributes.package = if thepackage then thepackage else Table.INHERIT
      
    @setPhpName = (name) ->
      name = name.trim() if typeof name is "string"
      _attributes.phpName = if name then name else null
      
    @setPhpNamingMethod = (method = Table.PhpNamingMethod.UNDERSCORE) ->
      if method in Table.PhpNamingMethod.toArray() then _method = method else Table.INHERIT
      
    @setReadOnly = (bool = true) ->
      _attributes.readOnly = if bool then true else false
      
    @setReloadOnInsert = (bool = true) ->
      _attributes.realoadOnInsert = if bool then true else false
      
    @setReloadOnUpdate = (bool = true) ->
      _attributes.realoadOnUpdate = if bool then true else false
      
    @setSchema = (schema) ->
      schema = schema.trim if typeof schema is "string"
      _attributes.schema = if schema then schema else Table.INHERIT
      
    @setSkipSql = (bool = true) ->
      _attributes.skipSql = if bool then true else false
      
    @setTreeMode = (treeMode) ->
      _attributes.treeMode = if treeMode in Table.TreeMode.toArray() then treeMode else null
      
    @setName name
\end{lstlisting}

\clearpage
\section{CoffeeScript Inheritance --- jQuery extension}
This code example shows how namespaces were implemented in the UI library. It demonstrates the work-around for jQuery's lack of OO design which can be seen in the \texttt{decorate()} function which adds its overriding functions to the \texttt{this} (\$with is effectively \texttt{this} in the decorate function). These functions are the jQuery functions such as html(), text() and prepend().

It has more prototypical functions than the previous example as it requires no privileged functions.

This example has been subjected to line wrapping. Whitespace in Coffee\-Script is significant and, unfortunately, the wrapping here would not constitute well-formed Coffee\-Script code.

\begin{lstlisting}
$ = jQuery
window.Bootstrap or=
  Components:
    Alert: class Alert extends jQuery
      constructor: (text) ->
        Alert.decorate $("<div>"), this
        $button = $ "<button>", type: "button", "data-dismiss": "alert"
        $button.addClass "close"
        $button.text "x"
        this.append $button
        
      alertBlock: (alertBlock = true) ->
        this.addClass "alert"
        if alertBlock
          $this.addClass "alert-block"
        else
          $this.removeClass "alert-block"
          
      setContext: (context = "warning", message) ->
        contexts = ["error", "info", "success", "warning"]
        this.removeClass "alert-#{contexts.join " alert-"}"
        if context not in contexts then context = "warning"
        this.addClass "alert alert-#{context}"
        this.html message if message?
        this
          
      danger: (message) -> this.setContext "error", message
        
      @decorate: ($item, $with = new Alert()) ->
        if $item not instanceof Object then $item = $ $item
        $.extend $with, $item
        $with.error = (message) -> Alert.prototype.error.call this, message
        $with.html = (html) -> Alert.prototype.html.call this, html
        $with.prepend = () -> Alert.prototype.prepend.apply this, arguments
        $with.remove = () -> Alert.prototype.remove.call this
        $with.text = (text) -> Alert.prototype.text.call this, text
        $with.addClass "fade in"
        Alert.prototype.setContext.call $with, "warning"
      
      error: (message) -> this.danger message
      
      html: (html) ->
        $button = this.children(".close").first()
        $button.detach()
        if typeof html is "function" then html = html.call this, 0, this.get(0)
        this.empty().append $button, $ $.parseHTML html
      
      info: (message) -> this.setContext "info", message
      
      prepend: () ->
        item = []
        if arguments.length > 1
          item.push arguments[arg] for arg of arguments
        else if typeof arguments[0] is "function" then item = aguments[0].call this, 0, this.get(0)
        else item = arguments[0]
        $button = this.children(".close").first()
        $siblings = this.detach(".close:first").contents()
        this.empty().append $button, item, $siblings
        
      remove: () ->
        if typeof this.alert is "function"
          this.alert("close") # for bootstrap-alert.js
        else
          this.remove()
      
      success: (message) -> this.setContext "success", message
      
      text: (text) ->
        $button = this.children(".close").first()
        $button.detach()
        if typeof text is "function" then text = text.call this, 0, this.text()
        this.empty().append $button, document.createTextNode text
      
      warning: (message) -> this.setContext "warning", message
\end{lstlisting}

